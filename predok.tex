\thispagestyle{empty}
\noindent
\strednp{
\NazovUniverzity\\ 
\NazovFakulty
}
\vfill
\strednp{
\NazovDiela
%% \centerline{\PodazovDiela}
\mbox{}\\
\bigskip
\TypPrace
}
\vfill
\strednp{\rok \hfill{\bf \autor}}
\newpage

\thispagestyle{empty}
\noindent
\strednp{\NazovUniverzity\\ \NazovFakulty}
\vfill
\strednp{\NazovDiela
%% \centerline{\PodazovDiela}
\mbox{}\\
\bigskip
\TypPrace
}
\vfill
\begin{tabular}{ l l }
\textbf{Študijný program:} & \program\\
\textbf{Študijný odbor:} & \cisloOdboru\ \odbor\\
\textbf{Školiace pracovisko:} & \katedra\\
\textbf{Vedúci práce:} &  \veduci
\end{tabular}
\bigskip\\
\bigskip\\
\bigskip\\
\bigskip\\
\strednp{\miestoRok \hfill{\bf \autor}}
\newpage

\noindent
~\vfill

\section*{Poďakovanie}
Veľká vďaka za to, že mi pomohli patrí môjmu školiteľovi Kubovi Kováčovi a 
mojim rodičom.\\
\bigskip\\
\newpage

\chapter*{Abstrakt}
Táto práca sa zaoberá vizualizáciou dátových štruktúr. Nadväzuje na bakalársku 
prácu Jakuba Kováča (2007) a rozvíja kompiláciu o ďalšie dátové štruktúry. V 
prvej časti sme tieto dátové štruktúry, konkrétne union-find, lexikografický 
strom a sufixový strom, definovali a konkrétne algoritmy popísali. V druhej 
časti sme spravili návrh aplikácie a popísali implementáciu. Práca je 
prehľadom dátových štruktúr s priloženým Java appletom na CD ako pomôckou pri 
vizualizácií.\\
Kľúčové slová: vizualizácia, ADŠ, algoritmy a dátové štruktúry, union-find, 
stringológia, lexikografický strom, trie, sufixový strom.

\newpage

\chapter*{Abstract}
This work is about visualization of algorithms and data structures. Continues 
the bachelor thesis of Jakub Kováč (2007) and extends the compilation by 
another data structures. In the first part are the structures, namely 
union-find, lexicographical tree (trie) and suffix tree, defined and the 
algorithms described. The second part is a description and a implementation of 
the software. The work is an overview of data structures with a software CD 
included.\\
Keywords: visualization, ADS, algorithms and data structures, union-find, 
stringology, loxicographical tree, trie, suffix tree.
\newpage

\mbox{}
\newpage

\tableofcontents\newpage
%\listoffigures
%\listoftables
